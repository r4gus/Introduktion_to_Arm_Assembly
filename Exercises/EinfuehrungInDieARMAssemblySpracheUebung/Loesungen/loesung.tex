\documentclass[12pt]{article}
\parindent0em
\parskip 1ex plus 0.4ex minus 0.4ex

\usepackage[a4paper,vmargin=30mm,hmargin=25mm]{geometry}
\usepackage{polyglossia}
\setdefaultlanguage{german}
\usepackage{fontspec}
\usepackage{lipsum}
\usepackage{xcolor}
\usepackage{listings}

\definecolor{lstbackground}{rgb}{0.95,0.95,1}      % hellgruener Rahmen

\newcommand{\instruction}[1]{\textbf{\texttt{\color{blue}#1}}}

\definecolor{codegreen}{rgb}{0,0.3,0}
\definecolor{codegray}{rgb}{0.5,0.5,0.5}
\definecolor{codepurple}{rgb}{0.58,0,0.82}
\definecolor{backcolour}{rgb}{0.95,0.95,0.92}
\definecolor{blue}{RGB}{0,0,255}
\definecolor{gray}{RGB}{224,224,224}
 
\lstdefinestyle{mystyle}{
    backgroundcolor=\color{backcolour},   
    commentstyle=\color{codegreen},
    keywordstyle=\color{blue},
    numberstyle=\tiny\color{codegray},
    stringstyle=\color{codepurple},
    basicstyle=\ttfamily\footnotesize,
    breakatwhitespace=false,         
    breaklines=true,                 
    captionpos=b,                    
    keepspaces=true,                 
    %numbers=left,                    
    numbersep=5pt,                  
    showspaces=false,                
    showstringspaces=false,
    showtabs=false,                  
    tabsize=4,
    morekeywords={add, adc, sub, sbc, mul, mla, 
    and, orr, eor, bic, cmp, cmn, tst, teq,
    mov, mvn, adds, adcs, subs, sbcs,
    muls, mlas, ands, orrs, eors, bics, cmp, cmn, tst, teq,
    mov, mvn, b, bl, blx, beq, bne, blt, bgt, bcs, bcc, bmi, bpl,
    bvs, bvc, bhi, bls, bge, ble,addeq,addne,addcs,addcc,addmi,addpl,addvs,addvc,
    addhi,addls,addge,addlt,addgt,addle,adceq,adcne,adccs,adccc,adcmi,adcpl,adcvs,
    adcvc,adchi,adcls,adcge,adclt,adcgt,adcle,subeq,subne,subcs,subcc,submi,subpl,
    subvs,subvc,subhi,subls,subge,sublt,subgt,suble,sbceq,sbcne,sbccs,sbccc,sbcmi,
    sbcpl,sbcvs,sbcvc,sbchi,sbcls,sbcge,sbclt,sbcgt,sbcle,muleq,mulne,mulcs,mulcc,
    mulmi,mulpl,mulvs,mulvc,mulhi,mulls,mulge,mullt,mulgt,mulle,mlaeq,mlane,mlacs,
    mlacc,mlami,mlapl,mlavs,mlavc,mlahi,mlals,mlage,mlalt,mlagt,mlale,andeq,andne,
    andcs,andcc,andmi,andpl,andvs,andvc,andhi,andls,andge,andlt,andgt,andle,orreq,
    orrne,orrcs,orrcc,orrmi,orrpl,orrvs,orrvc,orrhi,orrls,orrge,orrlt,orrgt,orrle,
    eoreq,eorne,eorcs,eorcc,eormi,eorpl,eorvs,eorvc,eorhi,eorls,eorge,eorlt,eorgt,
    eorle,biceq,bicne,biccs,biccc,bicmi,bicpl,bicvs,bicvc,bichi,bicls,bicge,biclt,
    bicgt,bicle,cmpeq,cmpne,cmpcs,cmpcc,cmpmi,cmppl,cmpvs,cmpvc,cmphi,cmpls,cmpge,
    cmplt,cmpgt,cmple,cmneq,cmnne,cmncs,cmncc,cmnmi,cmnpl,cmnvs,cmnvc,cmnhi,cmnls,
    cmnge,cmnlt,cmngt,cmnle,tsteq,tstne,tstcs,tstcc,tstmi,tstpl,tstvs,tstvc,tsthi,
    tstls,tstge,tstlt,tstgt,tstle,teqeq,teqne,teqcs,teqcc,teqmi,teqpl,teqvs,teqvc,
    teqhi,teqls,teqge,teqlt,teqgt,teqle,moveq,movne,movcs,movcc,movmi,movpl,movvs,
    movvc,movhi,movls,movge,movlt,movgt,movle,mvneq,mvnne,mvncs,mvncc,mvnmi,mvnpl,
    mvnvs,mvnvc,mvnhi,mvnls,mvnge,mvnlt,mvngt,mvnle,ADD,ADDS,ADDEQ,ADDEQS,ADDNE,ADDNES,ADDCS,ADDCSS,ADDCC,ADDCCS,ADDMI,ADDMIS,ADDPL,ADDPLS,ADDVS,ADDVSS,ADDVC,ADDVCS,ADDHI,ADDHIS,ADDLS,ADDLSS,ADDGE,ADDGES,ADDLT,ADDLTS,ADDGT,ADDGTS,ADDLE,ADDLES,ADC,ADCS,ADCEQ,ADCEQS,ADCNE,ADCNES,ADCCS,ADCCSS,ADCCC,ADCCCS,ADCMI,ADCMIS,ADCPL,ADCPLS,ADCVS,ADCVSS,ADCVC,ADCVCS,ADCHI,ADCHIS,ADCLS,ADCLSS,ADCGE,ADCGES,ADCLT,ADCLTS,ADCGT,ADCGTS,ADCLE,ADCLES,SUB,SUBS,SUBEQ,SUBEQS,SUBNE,SUBNES,SUBCS,SUBCSS,SUBCC,SUBCCS,SUBMI,SUBMIS,SUBPL,SUBPLS,SUBVS,SUBVSS,SUBVC,SUBVCS,SUBHI,SUBHIS,SUBLS,SUBLSS,SUBGE,SUBGES,SUBLT,SUBLTS,SUBGT,SUBGTS,SUBLE,SUBLES,SBC,SBCS,SBCEQ,SBCEQS,SBCNE,SBCNES,SBCCS,SBCCSS,SBCCC,SBCCCS,SBCMI,SBCMIS,SBCPL,SBCPLS,SBCVS,SBCVSS,SBCVC,SBCVCS,SBCHI,SBCHIS,SBCLS,SBCLSS,SBCGE,SBCGES,SBCLT,SBCLTS,SBCGT,SBCGTS,SBCLE,SBCLES,MUL,MULS,MULEQ,MULEQS,MULNE,MULNES,MULCS,MULCSS,MULCC,MULCCS,MULMI,MULMIS,MULPL,MULPLS,MULVS,MULVSS,MULVC,MULVCS,MULHI,MULHIS,MULLS,MULLSS,MULGE,MULGES,MULLT,MULLTS,MULGT,MULGTS,MULLE,MULLES,MLA,MLAS,MLAEQ,MLAEQS,MLANE,MLANES,MLACS,MLACSS,MLACC,MLACCS,MLAMI,MLAMIS,MLAPL,MLAPLS,MLAVS,MLAVSS,MLAVC,MLAVCS,MLAHI,MLAHIS,MLALS,MLALSS,MLAGE,MLAGES,MLALT,MLALTS,MLAGT,MLAGTS,MLALE,MLALES,AND,ANDS,ANDEQ,ANDEQS,ANDNE,ANDNES,ANDCS,ANDCSS,ANDCC,ANDCCS,ANDMI,ANDMIS,ANDPL,ANDPLS,ANDVS,ANDVSS,ANDVC,ANDVCS,ANDHI,ANDHIS,ANDLS,ANDLSS,ANDGE,ANDGES,ANDLT,ANDLTS,ANDGT,ANDGTS,ANDLE,ANDLES,ORR,ORRS,ORREQ,ORREQS,ORRNE,ORRNES,ORRCS,ORRCSS,ORRCC,ORRCCS,ORRMI,ORRMIS,ORRPL,ORRPLS,ORRVS,ORRVSS,ORRVC,ORRVCS,ORRHI,ORRHIS,ORRLS,ORRLSS,ORRGE,ORRGES,ORRLT,ORRLTS,ORRGT,ORRGTS,ORRLE,ORRLES,EOR,EORS,EOREQ,EOREQS,EORNE,EORNES,EORCS,EORCSS,EORCC,EORCCS,EORMI,EORMIS,EORPL,EORPLS,EORVS,EORVSS,EORVC,EORVCS,EORHI,EORHIS,EORLS,EORLSS,EORGE,EORGES,EORLT,EORLTS,EORGT,EORGTS,EORLE,EORLES,BIC,BICS,BICEQ,BICEQS,BICNE,BICNES,BICCS,BICCSS,BICCC,BICCCS,BICMI,BICMIS,BICPL,BICPLS,BICVS,BICVSS,BICVC,BICVCS,BICHI,BICHIS,BICLS,BICLSS,BICGE,BICGES,BICLT,BICLTS,BICGT,BICGTS,BICLE,BICLES,CMP,CMPS,CMPEQ,CMPEQS,CMPNE,CMPNES,CMPCS,CMPCSS,CMPCC,CMPCCS,CMPMI,CMPMIS,CMPPL,CMPPLS,CMPVS,CMPVSS,CMPVC,CMPVCS,CMPHI,CMPHIS,CMPLS,CMPLSS,CMPGE,CMPGES,CMPLT,CMPLTS,CMPGT,CMPGTS,CMPLE,CMPLES,CMN,CMNS,CMNEQ,CMNEQS,CMNNE,CMNNES,CMNCS,CMNCSS,CMNCC,CMNCCS,CMNMI,CMNMIS,CMNPL,CMNPLS,CMNVS,CMNVSS,CMNVC,CMNVCS,CMNHI,CMNHIS,CMNLS,CMNLSS,CMNGE,CMNGES,CMNLT,CMNLTS,CMNGT,CMNGTS,CMNLE,CMNLES,TST,TSTS,TSTEQ,TSTEQS,TSTNE,TSTNES,TSTCS,TSTCSS,TSTCC,TSTCCS,TSTMI,TSTMIS,TSTPL,TSTPLS,TSTVS,TSTVSS,TSTVC,TSTVCS,TSTHI,TSTHIS,TSTLS,TSTLSS,TSTGE,TSTGES,TSTLT,TSTLTS,TSTGT,TSTGTS,TSTLE,TSTLES,TEQ,TEQS,TEQEQ,TEQEQS,TEQNE,TEQNES,TEQCS,TEQCSS,TEQCC,TEQCCS,TEQMI,TEQMIS,TEQPL,TEQPLS,TEQVS,TEQVSS,TEQVC,TEQVCS,TEQHI,TEQHIS,TEQLS,TEQLSS,TEQGE,TEQGES,TEQLT,TEQLTS,TEQGT,TEQGTS,TEQLE,TEQLES,MOV,MOVS,MOVEQ,MOVEQS,MOVNE,MOVNES,MOVCS,MOVCSS,MOVCC,MOVCCS,MOVMI,MOVMIS,MOVPL,MOVPLS,MOVVS,MOVVSS,MOVVC,MOVVCS,MOVHI,MOVHIS,MOVLS,MOVLSS,MOVGE,MOVGES,MOVLT,MOVLTS,MOVGT,MOVGTS,MOVLE,MOVLES,MVN,MVNS,MVNEQ,MVNEQS,MVNNE,MVNNES,MVNCS,MVNCSS,MVNCC,MVNCCS,MVNMI,MVNMIS,MVNPL,MVNPLS,MVNVS,MVNVSS,MVNVC,MVNVCS,MVNHI,MVNHIS,MVNLS,MVNLSS,MVNGE,MVNGES,MVNLT,MVNLTS,MVNGT,MVNGTS,MVNLE,MVNLES,LDR,LDRS,LDREQ,LDREQS,LDRNE,LDRNES,LDRCS,LDRCSS,LDRCC,LDRCCS,LDRMI,LDRMIS,LDRPL,LDRPLS,LDRVS,LDRVSS,LDRVC,LDRVCS,LDRHI,LDRHIS,LDRLS,LDRLSS,LDRGE,LDRGES,LDRLT,LDRLTS,LDRGT,LDRGTS,LDRLE,LDRLES,STR,STRS,STREQ,STREQS,STRNE,STRNES,STRCS,STRCSS,STRCC,STRCCS,STRMI,STRMIS,STRPL,STRPLS,STRVS,STRVSS,STRVC,STRVCS,STRHI,STRHIS,STRLS,STRLSS,STRGE,STRGES,STRLT,STRLTS,STRGT,STRGTS,STRLE,STRLES,LDM,LDMS,LDMEQ,LDMEQS,LDMNE,LDMNES,LDMCS,LDMCSS,LDMCC,LDMCCS,LDMMI,LDMMIS,LDMPL,LDMPLS,LDMVS,LDMVSS,LDMVC,LDMVCS,LDMHI,LDMHIS,LDMLS,LDMLSS,LDMGE,LDMGES,LDMLT,LDMLTS,LDMGT,LDMGTS,LDMLE,LDMLES,STM,STMS,STMEQ,STMEQS,STMNE,STMNES,STMCS,STMCSS,STMCC,STMCCS,STMMI,STMMIS,STMPL,STMPLS,STMVS,STMVSS,STMVC,STMVCS,STMHI,STMHIS,STMLS,STMLSS,STMGE,STMGES,STMLT,STMLTS,STMGT,STMGTS,STMLE,STMLES,LDMDB,LDMDBS,LDMDBEQ,LDMDBEQS,LDMDBNE,LDMDBNES,LDMDBCS,LDMDBCSS,LDMDBCC,LDMDBCCS,LDMDBMI,LDMDBMIS,LDMDBPL,LDMDBPLS,LDMDBVS,LDMDBVSS,LDMDBVC,LDMDBVCS,LDMDBHI,LDMDBHIS,LDMDBLS,LDMDBLSS,LDMDBGE,LDMDBGES,LDMDBLT,LDMDBLTS,LDMDBGT,LDMDBGTS,LDMDBLE,LDMDBLES,LDMDA,LDMDAS,LDMDAEQ,LDMDAEQS,LDMDANE,LDMDANES,LDMDACS,LDMDACSS,LDMDACC,LDMDACCS,LDMDAMI,LDMDAMIS,LDMDAPL,LDMDAPLS,LDMDAVS,LDMDAVSS,LDMDAVC,LDMDAVCS,LDMDAHI,LDMDAHIS,LDMDALS,LDMDALSS,LDMDAGE,LDMDAGES,LDMDALT,LDMDALTS,LDMDAGT,LDMDAGTS,LDMDALE,LDMDALES,LDMIB,LDMIBS,LDMIBEQ,LDMIBEQS,LDMIBNE,LDMIBNES,LDMIBCS,LDMIBCSS,LDMIBCC,LDMIBCCS,LDMIBMI,LDMIBMIS,LDMIBPL,LDMIBPLS,LDMIBVS,LDMIBVSS,LDMIBVC,LDMIBVCS,LDMIBHI,LDMIBHIS,LDMIBLS,LDMIBLSS,LDMIBGE,LDMIBGES,LDMIBLT,LDMIBLTS,LDMIBGT,LDMIBGTS,LDMIBLE,LDMIBLES,LDMIA,LDMIAS,LDMIAEQ,LDMIAEQS,LDMIANE,LDMIANES,LDMIACS,LDMIACSS,LDMIACC,LDMIACCS,LDMIAMI,LDMIAMIS,LDMIAPL,LDMIAPLS,LDMIAVS,LDMIAVSS,LDMIAVC,LDMIAVCS,LDMIAHI,LDMIAHIS,LDMIALS,LDMIALSS,LDMIAGE,LDMIAGES,LDMIALT,LDMIALTS,LDMIAGT,LDMIAGTS,LDMIALE,LDMIALES,STMDB,STMDBS,STMDBEQ,STMDBEQS,STMDBNE,STMDBNES,STMDBCS,STMDBCSS,STMDBCC,STMDBCCS,STMDBMI,STMDBMIS,STMDBPL,STMDBPLS,STMDBVS,STMDBVSS,STMDBVC,STMDBVCS,STMDBHI,STMDBHIS,STMDBLS,STMDBLSS,STMDBGE,STMDBGES,STMDBLT,STMDBLTS,STMDBGT,STMDBGTS,STMDBLE,STMDBLES,STMDA,STMDAS,STMDAEQ,STMDAEQS,STMDANE,STMDANES,STMDACS,STMDACSS,STMDACC,STMDACCS,STMDAMI,STMDAMIS,STMDAPL,STMDAPLS,STMDAVS,STMDAVSS,STMDAVC,STMDAVCS,STMDAHI,STMDAHIS,STMDALS,STMDALSS,STMDAGE,STMDAGES,STMDALT,STMDALTS,STMDAGT,STMDAGTS,STMDALE,STMDALES,STMIB,STMIBS,STMIBEQ,STMIBEQS,STMIBNE,STMIBNES,STMIBCS,STMIBCSS,STMIBCC,STMIBCCS,STMIBMI,STMIBMIS,STMIBPL,STMIBPLS,STMIBVS,STMIBVSS,STMIBVC,STMIBVCS,STMIBHI,STMIBHIS,STMIBLS,STMIBLSS,STMIBGE,STMIBGES,STMIBLT,STMIBLTS,STMIBGT,STMIBGTS,STMIBLE,STMIBLES,STMIA,STMIAS,STMIAEQ,STMIAEQS,STMIANE,STMIANES,STMIACS,STMIACSS,STMIACC,STMIACCS,STMIAMI,STMIAMIS,STMIAPL,STMIAPLS,STMIAVS,STMIAVSS,STMIAVC,STMIAVCS,STMIAHI,STMIAHIS,STMIALS,STMIALSS,STMIAGE,STMIAGES,STMIALT,STMIALTS,STMIAGT,STMIAGTS,STMIALE,STMIALES,ldreq,ldrne,ldrcs,ldrcc,ldrmi,ldrpl,ldrvs,ldrvc,ldrhi,ldrls,ldrge,ldrlt,ldrgt,ldrle,streq,strne,strcs,strcc,strmi,strpl,strvs,strvc,strhi,strls,strge,strlt,strgt,strle,ldmeq,ldmne,ldmcs,ldmcc,ldmmi,ldmpl,ldmvs,ldmvc,ldmhi,ldmls,ldmge,ldmlt,ldmgt,ldmle,stmeq,stmne,stmcs,stmcc,stmmi,stmpl,stmvs,stmvc,stmhi,stmls,stmge,stmlt,stmgt,stmle,ldmiaeq,ldmiane,ldmiacs,ldmiacc,ldmiami,ldmiapl,ldmiavs,ldmiavc,ldmiahi,ldmials,ldmiage,ldmialt,ldmiagt,ldmiale,ldmibeq,ldmibne,ldmibcs,ldmibcc,ldmibmi,ldmibpl,ldmibvs,ldmibvc,ldmibhi,ldmibls,ldmibge,ldmiblt,ldmibgt,ldmible,ldmdaeq,ldmdane,ldmdacs,ldmdacc,ldmdami,ldmdapl,ldmdavs,ldmdavc,ldmdahi,ldmdals,ldmdage,ldmdalt,ldmdagt,ldmdale,ldmdbeq,ldmdbne,ldmdbcs,ldmdbcc,ldmdbmi,ldmdbpl,ldmdbvs,ldmdbvc,ldmdbhi,ldmdbls,ldmdbge,ldmdblt,ldmdbgt,ldmdble,stmiaeq,stmiane,stmiacs,stmiacc,stmiami,stmiapl,stmiavs,stmiavc,stmiahi,stmials,stmiage,stmialt,stmiagt,stmiale,stmibeq,stmibne,stmibcs,stmibcc,stmibmi,stmibpl,stmibvs,stmibvc,stmibhi,stmibls,stmibge,stmiblt,stmibgt,stmible,stmdbeq,stmdbne,stmdbcs,stmdbcc,stmdbmi,stmdbpl,stmdbvs,stmdbvc,stmdbhi,stmdbls,stmdbge,stmdblt,stmdbgt,stmdble,stmdaeq,stmdane,stmdacs,stmdacc,stmdami,stmdapl,stmdavs,stmdavc,stmdahi,stmdals,stmdage,stmdalt,stmdagt,stmdale,
ldr,str,ldm,stm, adr, svc, SVC, swi, SWI, bx, blx, BLX, BX, push, pop, PUSH,
POP, umull},
  literate=%
  {Ö}{{\"O}}1
  {Ä}{{\"A}}1
  {Ü}{{\"U}}1
  {ß}{{\ss}}1
  {ü}{{\"u}}1
  {ä}{{\"a}}1
  {ö}{{\"o}}1
  {~}{{\textasciitilde}}1
}
 
\lstset{style=mystyle}

\begin{document}

\begin{center}
  \textbf{\LARGE Name der Vorlesung} \\[1ex]%
  \textbf{\Large Aufgaben zu Einführung in die ARM Assembler Sprache}\\[2ex] %
  
\end{center}

% ****************************************************************************
\section{Aufgabe 1 - Verständniss}
% ****************************************************************************

% ----------------------------------------------------------------------------
\subsection{Allgemeines}
% ----------------------------------------------------------------------------

\begin{itemize}
 \item Welches Speichermodell verwenden die ARM Architekturen ?
     \begin{itemize}
        \item Load-Store Speichermodell.
     \end{itemize}
 \item Welche Wortbreite besitzt eine Instruktion ?
     \begin{itemize}
        \item 32 Bit unter ARM, 16 Bit unter Thumb.
     \end{itemize}
 \item Welche der ARM Instruktionen verwenden Erweiterungsworte ?
     \begin{itemize}
         \item Es gibt \textbf{keine} Erweiterungsworte unter ARM.
     \end{itemize}
 \item Wie werden Instruktionen verarbeitet ?
     \begin{itemize}
         \item Mittels Pipelining.
     \end{itemize}
 \item Angenommen zum Zeitpunkt n wird eine ADD Instruktion, an der Adresse
     \texttt{0xAFF6}, tatsächlich ausgeführt (execute Stufe), an welcher Adresse befindet sich
        der \texttt{PC}, wenn sich der Prozessor im ARM Modus befindet?
     \begin{itemize}
         \item PC = 0xAFFE
     \end{itemize}
 \item Was ist die Tool Chain? Geben pro Programmierwerkzeug eine kurze
     Beschreibung.
     \begin{description}
         \item [Assembler] Erzeugt aus einer Assembler-Source-File eine
             Objektdatei (relocation file).
         \item [Linker] Verlinkt (mehrere) Objektdateien zu einem ausführbaren
             Programm.
        \item [Debugger] Programm zur Analyse und Kontrolle einer ausführbaren
            Datei.
     \end{description}
 \item Nennen Sie drei mögliche Sektionen, in die eine ausführbare Datei
     untergliedert ist und beschreiben Sie diese kurz.
        \begin{description}
            \item [text]
            \item [data]
            \item [bss]
            \item [rodata]
        \end{description}
 \item Zeichnen Sie das typische Speicher-Layout eines Prozesses unter Linux.
\end{itemize}

% ----------------------------------------------------------------------------
\subsection{Bits and Bytes}
% ----------------------------------------------------------------------------

\begin{itemize}
    \item Welche Größe (in Bit) besitzen: \texttt{BYTE, HALF WORD, WORD, DOUBLE WORD}?
    \item Welche Endianness wird unter ARM genutzt?
    \item Beschreiben Sie \texttt{Little Endian} in einem Satz.
\end{itemize}

% ----------------------------------------------------------------------------
\subsection{Register}
% ----------------------------------------------------------------------------

\begin{itemize}
    \item Was sind \texttt{Callee Save Register}?
    \item Welches Register (Rx) stellt den Program Counter dar?
    \item Was ist der Zweck des Link Registers? Was ist in dieser Hinsicht der
        Unterschied zu x86?
    \item Nennen Sie die Flags, die für bedingte Instruktionen zur verfügung stehen.
    \item Nennen Sie zwei Prozessor Modi und beschreiben Sie diese kurz.
\end{itemize}

% ***************************************************************************
\section{Data Processing Instructions}
% ***************************************************************************
Für die nachfolgenden Aufgaben wird ein Arm Computer, wie z.B. Raspberry Pi,
oder ein Emulator, wie z.B. Quemu, benötigt, sowie ein Linux Betriebssystem.

Die Beschreibungen beziehen sich auf eine ARM Entwicklungsumgebung
bestehend aus eine \texttt{Raspberry Pi 3} auf dem \texttt{Raspbian} installiert
wurde.

% ----------------------------------------------------------------------------
\subsection{Einrichten der Entwicklungsumgebung}
% ----------------------------------------------------------------------------
Sie können den Raspberry Pi mittels HDMI und USB direkt mit einem Monitor, sowie
Tastatur verbinden. 

Als Alternative können Sie sich auch mittels \texttt{SSH},
von einem anderen Computer aus, mit dem Raspberry Pi verbinden. Dafür müssen Sie
sich im selben Netzwerk befinden. Für den Verbindungsaufbau benötigen Sie
weiterhin die IP-Adresse des Gerätes. Angenommen Sie möchten sich als Nutzer
\texttt{pi} auf dem Gerät mit der IP-Adresse \texttt{192.168.62.255} einlogen,
dann können Sie dies über die Konsole mit folgendem Befehl bewerkstelligen:

\begin{lstlisting}
ssh pi@192.168.62.255
\end{lstlisting}


% ----------------------------------------------------------------------------
\subsection{Bearbeitung der Aufgaben}
% ----------------------------------------------------------------------------
Für die Bearbeitung der Aufgaben steht Ihnen eine Template Datei namens
\texttt{template.s} zur verfügung. Diese kann verwendet werden um die einzelnen
Aufgaben zu bearbeiten.

In der \texttt{main} Subroutine befindet sich bereits Beispielcode, der ersetzt
werden kann. Außerdem können natürlich weitere Funktionen definiert werden.

Sie können Ihren Source Code mithilfe des Compilerscripts \texttt{compiel.sh}
compilieren, dabei wird der Code von \texttt{debug.c} mit verlinkt, der Ihnen
die Möglichkeit bietet, Ihre Resultate auf der Kommandozeile auszugeben.

\begin{lstlisting}
./compile.sh template.s
\end{lstlisting}

Um Ihr ergebnis auszugeben, können Sie die \texttt{info()} funktion aufrufen.
Diese erwartet als erstes Argument das ausgabe Format, d.h.
\texttt{Hexadezimal, Oktal, vorzeichenbehafteter Integer, vorzeichenloser
Integer} und als zweites Argument den auszugebenden Wert.

\begin{lstlisting}
void info(const int func, const int val);
\end{lstlisting}

Angenommen es wurde eine simple Additionsfunktion implementiert und das Ergebnis
soll nun als Hexadezimalwert ausgegeben werden, dann kann \texttt{info} wie
folgt aufgerufen werden.

\begin{lstlisting}
Vorbedingung: Der auszugebende Wert befindet sich in R0

mov r1, r0   @ pass value as second argument
mov r0, #HEX @ specify output format
bl  info     @ call info
\end{lstlisting}


% ----------------------------------------------------------------------------
\subsection{Dividieren mal anders}
% ----------------------------------------------------------------------------
\subsubsection{a)}
Auf modernen Systemen ist ein Dividierwerk standard, mit dem ein Zahl durch eine
andere, mittels Hardware, geteilt werden kann. Die Armv-8 Architektur bietet
die Instruktionen \texttt{UDIV} und \texttt{SDIV} um vorzeichenlose bzw.
vorzeichenbehaftete Zahlen zu dividieren. Ältere Architekturen besitzten diese
Möglichkeit nicht. Dort werden Divisionen in Software, d.h. durch eine Funktion, realisiert. 
Solche Dividieralgorithmen sind jedoch sehr ineffizient und benötigen viele
Taktzyklen.

Berechnen Sie \texttt{(7 * 5) / 3 + 21} mit so wenigen Instruktionen wie
möglich, ohne die Verwendung eines Multiplikations- oder Dividierbefehls und
ohne die einzelnene Zahlen im Vorraus zusammenzufassen.

\subsubsection{b)}
Für einen Zyklischen Buffer der größe \texttt{buffer\_size} der mittels
\texttt{offset} indiziert wird, könnte eine erhöhung um \texttt{n} Wörter wie
folgt aussehen:

\begin{lstlisting}
offset = (offset + n) % buffer_size;
\end{lstlisting}

Angenommen der obige Ausdruck benötigt 50 Taktzyklen zur berechnung (aufgrund
der Modulo Rechnung), geben Sie
eine, ihrer Meinung nach, effizientere Lösung an. Implementieren Sie ihre Lösung
in Assembly. Wie viele effektive Taktzyklen benötigt Ihre Implementierung?

















\end{document}

%%% Local Variables: 
%%% TeX-PDF-mode: t
%%% TeX-master: t
%%% coding: utf-8
%%% mode: latex
%%% TeX-engine: xetex
%%% End: 
